%!TEX root = ../master.tex
\section{Einleitung} \label{chap:introduction}
Diese Vorlage dient als Richtlinie zur Erstellung von Seminararbeiten und bietet Beispiele für die Strukturierung der Arbeit. Für eine Vollständige Liste der Formatvorgaben und Richtlinien, siehe die \emph{Richtlinien für die Erstellung von Seminararbeiten} des Lehrstuhls für Wirtschaftsinformatik I und des Betriebswirtschaftlichen Instituts der Universität Stuttgart.
\section{Überschrift}
Zwischen Überschriften gehört Text.

\subsection{Unterüberschrift}
Zwischen Unterüberschriften gehört ebenfalls Text.

\subsubsection{Unterunterüberschrift}
Es sollten nur dann nummerierten Überschriften verwendet werden, wenn es auf der jeweiligen Ebene mindestens zwei Einträge gibt (kein 2.1 ohne 2.2, kein 2.1.1 ohne 2.1.2).

\subsubsection{Unterüberschrift}
Zwischen Überschriften sollte Text stehen.

\subsection{Unterüberschrift}
Zwischen Überschriften sollte Text stehen.

\subsection{In-Text Zitation}

In seinem Experiment beschreibt \textcite[1-10]{Hu60} die...

Studien attestieren dabei einen erhöhten Effekt auf das vegetative Nervensystem \parencite[3]{Hu60}.

"Die Daten sind Aussagekräftig und im Einklang mit bestehenden Studien". \textcite[5]{Hu60}


